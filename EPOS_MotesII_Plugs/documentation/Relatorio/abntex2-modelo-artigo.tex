%% abtex2-modelo-artigo.tex, v<VERSION> laurocesar
%% Copyright 2012-2013 by abnTeX2 group at http://abntex2.googlecode.com/ 
%%
%% This work may be distributed and/or modified under the
%% conditions of the LaTeX Project Public License, either version 1.3
%% of this license or (at your option) any later version.
%% The latest version of this license is in
%%   http://www.latex-project.org/lppl.txt
%% and version 1.3 or later is part of all distributions of LaTeX
%% version 2005/12/01 or later.
%%
%% This work has the LPPL maintenance status `maintained'.
%% 
%% The Current Maintainer of this work is the abnTeX2 team, led
%% by Lauro César Araujo. Further information are available on 
%% http://abntex2.googlecode.com/
%%
%% This work consists of the files abntex2-modelo-artigo.tex and
%% abntex2-modelo-references.bib
%%

% ------------------------------------------------------------------------
% ------------------------------------------------------------------------
% abnTeX2: Modelo de Artigo Acadêmico em conformidade com
% ABNT NBR 6022:2003: Informação e documentação - Artigo em publicação 
% periódica científica impressa - Apresentação
% ------------------------------------------------------------------------
% ------------------------------------------------------------------------

% oneside = apenas frente
% twocolumn = artigo em duas colunas
\documentclass[article,11pt,oneside,a4paper,english,brazil]{abntex2}	

% ---
% PACOTES
% ---

% ---
% Pacotes fundamentais 
% ---
\usepackage{cmap}				% Mapear caracteres especiais no PDF
\usepackage{lmodern}			% Usa a fonte Latin Modern
\usepackage[T1]{fontenc}		% Selecao de codigos de fonte.
\usepackage[utf8]{inputenc}		% Codificacao do documento (conversão automática dos acentos)
\usepackage{indentfirst}		% Indenta o primeiro parágrafo de cada seção.
\usepackage{nomencl} 			% Lista de simbolos
\usepackage{color}				% Controle das cores
\usepackage{graphicx}			% Inclusão de gráficos

\usepackage{listings}

\definecolor{dkgreen}{rgb}{0,0.6,0}
\definecolor{gray}{rgb}{0.5,0.5,0.5}
\definecolor{mauve}{rgb}{0.58,0,0.82}

\lstset{frame=tb,
	language=Java,
	aboveskip=3mm,
	belowskip=3mm,
	showstringspaces=false,
	columns=flexible,
	basicstyle={\small\ttfamily},
	numbers=none,
	numberstyle=\tiny\color{gray},
	keywordstyle=\color{blue},
	commentstyle=\color{dkgreen},
	stringstyle=\color{mauve},
	breaklines=true,
	breakatwhitespace=true,
	tabsize=3
}

% ---
		
% ---
% Pacotes adicionais, usados apenas no âmbito do Modelo Canônico do abnteX2
% ---
\usepackage{lipsum}				% para geração de dummy text
% ---
		
% ---
% Pacotes de citações
% ---
\usepackage[brazilian,hyperpageref]{backref}	 % Paginas com as citações na bibl
\usepackage[alf]{abntex2cite}	% Citações padrão ABNT
% ---

% ---
% Configurações do pacote backref
% Usado sem a opção hyperpageref de backref
\renewcommand{\backrefpagesname}{Citado na(s) página(s):~}
% Texto padrão antes do número das páginas
\renewcommand{\backref}{}
% Define os textos da citação
\renewcommand*{\backrefalt}[4]{
	\ifcase #1 %
		Nenhuma citação no texto.%
	\or
		Citado na página #2.%
	\else
		Citado #1 vezes nas páginas #2.%
	\fi}%
% ---

% ---
% Informações de dados para CAPA e FOLHA DE ROSTO
% ---
\titulo{Controle integrado e distribuído de tomadas inteligentes}
\autor{Luca Fachini Campelli \and Marcio Monteiro \and  Rodrigo Pedro Marques}
\local{Brasil}
\data{Florianópolis, 2016}
% ---

% ---
% Configurações de aparência do PDF final

% alterando o aspecto da cor azul
\definecolor{blue}{RGB}{41,5,195}

% informações do PDF
\makeatletter
\hypersetup{
     	%pagebackref=true,
		pdftitle={\@title}, 
		pdfauthor={\@author},
    	pdfsubject={Modelo de artigo científico com abnTeX2},
	    pdfcreator={LaTeX with abnTeX2},
		pdfkeywords={abnt}{latex}{abntex}{abntex2}{atigo científico}, 
		colorlinks=true,       		% false: boxed links; true: colored links
    	linkcolor=blue,          	% color of internal links
    	citecolor=blue,        		% color of links to bibliography
    	filecolor=magenta,      		% color of file links
		urlcolor=blue,
		bookmarksdepth=4
}
\makeatother
% --- 

% ---
% compila o indice
% ---
\makeindex
% ---

% ---
% Altera as margens padrões
% ---
\setlrmarginsandblock{3cm}{2cm}{*}
\setulmarginsandblock{3cm}{2cm}{*}
\checkandfixthelayout
% ---

% --- 
% Espaçamentos entre linhas e parágrafos 
% --- 

% O tamanho do parágrafo é dado por:
\setlength{\parindent}{1.3cm}

% Controle do espaçamento entre um parágrafo e outro:
\setlength{\parskip}{0.2cm}  % tente também \onelineskip

% Espaçamento simples
\SingleSpacing

% ----
% Início do documento
% ----
\begin{document}

% Retira espaço extra obsoleto entre as frases.
\frenchspacing 

% ----------------------------------------------------------
% ELEMENTOS PRÉ-TEXTUAIS
% ----------------------------------------------------------

%---
%
% Se desejar escrever o artigo em duas colunas, descomente a linha abaixo
% e a linha com o texto ``FIM DE ARTIGO EM DUAS COLUNAS''.
% \twocolumn[    		% INICIO DE ARTIGO EM DUAS COLUNAS
%
%---
% página de titulo
\maketitle

% resumo em português
\begin{resumoumacoluna}
	aqui vai resumo
	
	\vspace{\onelineskip}
	
	\noindent
	\textbf{Palavras-chaves}: tomadas inteligentes, sistemas operacionais, eposmote2.
\end{resumoumacoluna}

% ]  				% FIM DE ARTIGO EM DUAS COLUNAS
% ---

% ----------------------------------------------------------
% ELEMENTOS TEXTUAIS
% ----------------------------------------------------------
\textual

% ----------------------------------------------------------
% Introdução
% ----------------------------------------------------------
\section{Introdução}

	Este trabalho foi realizado por Luca Campelli, Marcio Monteiro e Rodrigo Pedro Marques para a disciplina INE5412 - Sistemas Operacionais I, no semestre 2016.1. O objetivo principal deste trabalho é implementar o descobrimento automático de tomadas em uma rede onde elas possam tomar decisão sobre o consumo energético.
	
	Para a realização deste projeto, foram utilizados três placas EPOSMote II \cite{eposproject} que simularam o comportamento de tomadas inteligentes. Para que isto fosse possível, o projeto foi subdividido em partes. São elas: realizar a comunicação entre as placas via \textit{broadcast}, propor o algoritmo para consumo inteligente de energia, implementação deste algoritmo, validação e testes do algoritmo.
	
	Este relatório está organizado da seguinte forma: em \ref{sec:desenvolvimento} são apresentadas as etapas em maior detalhes relacionadas ao desenvolvimento deste projeto; em \ref{sec:validacaoetestes} são apresentados os testes realizados após o desenvolvimento; em \ref{sec:consideracoes} são apresentadas as considerações finais e conclusões em relação a este projeto; finalizando este relatório, na seção \ref{sec:trabalhosfuturos} apresentamos possíveis temas para serem discutidos posteriormente.
	
\section{Desenvolvimento} \label{sec:desenvolvimento}
	
	Para que o projeto pudesse ser desenvolvido, primeiramente foi necessário descobrir como era realizada a comunicação entre as placas. Para tal, foram realizados exemplos que constam na documentação do EPOSMote II sobre comunicação entre placas via \textit{broadcast}. No código a seguir é possível observar como foi implementado o código para alcançar este objetivo no projeto.
		
	\begin{lstlisting}
		int Mensageiro::receberViaNIC(){
			NIC nic;		
			NIC::Address src;
			unsigned char prot;
			infoTomadas meg;
			
			while(1){
				while(!nic.receive(&src, &prot,&meg,sizeof(meg)) >0);
				if(meg.address != 0){
					gerente->receberMensagem(meg);
				}
			meg.address = 0;
			}
			return 0;
		}	
		
		Mensageiro::Mensageiro(Gerente * gnt){
			gerente = gnt;
			Thread *thread;
			thread = new Thread( \&receberViaNIC );
		}
	\end{lstlisting}
	
	Como é possível observar, foi utilizada a função NIC, já presente na no EPOSMoteII. Ela é responsável por prover acesso à redes. 
	
	\begin{figure}[b]
		\centering
		\includegraphics[width=0.7\linewidth]{figuras/uml}
		\caption[Short]{UML do Projeto.}
		\label{fig:uml}
	\end{figure}
	
	A figura \ref{fig:uml} apresenta a modelagem em UML do projeto. A classe \textit{Tomada} representa uma tomada física, inteligente ou não. Neste projeto foram consideradas quatro tipos de tomadas: uma tomada simples, que liga e desliga; uma tomada com \textit{dimer}, que possibilita \textit{dimerizar (controlar)} a energia que passa por ela; uma tomada com sensor, onde é possível medir o seu consumo, colocar em modo de economia de energia, atribuir um limite de consumo, atribuir uma prioridade a ela e verificar o seu consumo mínimo e máximo já registrado; e uma tomada \textit{top} que possui todas as funções das tomadas citadas anteriormente. Cada tomada possui um \textit{Gerente Monitor} responsável por monitorar e atualizar a tomada, além de enviar e receber mensagens de outras tomadas no ambiente. Além disto, ele mantem o endereço de todas as tomadas no ambiente, guarda informações (previsão do consumo de energia, endereço da tomada, prioridade e tipo) da tomada monitorada por ele e também possui um \textit{Mensageiro} e um \textit{Previsor}. A classe \textit{Mensageiro} é responsável por receber mensagens via NIC ou UART. O \textit{Previsor} auxilia o gerente realizando as previsões de consumo da tomada gerenciada por ele. Esta classe pode fazer as previsões de consumo diárias, mensal e de todas as tomadas do ambiente. 

\section{Validação e Testes} \label{sec:validacaoetestes}


\section{Considerações Finais} \label{sec:consideracoes}


\section{Trabalhos Futuros} \label{sec:trabalhosfuturos}
% ----------------------------------------------------------
% ELEMENTOS PÓS-TEXTUAIS
% ----------------------------------------------------------
\postextual

% ----------------------------------------------------------
% Referências bibliográficas
% ----------------------------------------------------------
\bibliography{abntex2-modelo-references}

% ----------------------------------------------------------
% Glossário
% ----------------------------------------------------------
%
% Há diversas soluções prontas para glossário em LaTeX. 
% Consulte o manual do abnTeX2 para obter sugestões.
%
%\glossary

% ----------------------------------------------------------
% Apêndices
% ----------------------------------------------------------

% ---
% Inicia os apêndices
% ---
%\begin{apendicesenv}
%
%% ----------------------------------------------------------
%\chapter{Nullam elementum urna vel imperdiet sodales elit ipsum pharetra ligula
%ac pretium ante justo a nulla curabitur tristique arcu eu metus}
%% ----------------------------------------------------------
%\lipsum[55-57]
%
%\end{apendicesenv}
% ---

% ----------------------------------------------------------
% Anexos
% ----------------------------------------------------------
%\cftinserthook{toc}{AAA}
% ---
% Inicia os anexos
% ---
%\anexos
%\begin{anexosenv}
%
%% ---
%\chapter{Cras non urna sed feugiat cum sociis natoque penatibus et magnis dis
%parturient montes nascetur ridiculus mus}
%% ---
%
%\lipsum[31]
%
%\end{anexosenv}


% ---
% Título e resumo em língua estrangeira
% ---

% \twocolumn[    		% INICIO DE ARTIGO EM DUAS COLUNAS

% titulo em inglês
%\titulo{Canonical academic article model with \abnTeX}
%\emptythanks
%\maketitle

% resumo em português
%\renewcommand{\resumoname}{Abstract}
%\begin{resumoumacoluna}
% \begin{otherlanguage*}{english}
%   According to ABNT NBR 6022:2003, an abstract in foreign language is a back
%   matter mandatory element.
%
%   \vspace{\onelineskip}
% 
%   \noindent
%   \textbf{Key-words}: latex. abntex.
% \end{otherlanguage*}  
%\end{resumoumacoluna}

% ]  				% FIM DE ARTIGO EM DUAS COLUNAS
% ---

\end{document}